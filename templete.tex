\documentclass[11pt]{article}

% \usepackage[sort]{natbib}
\usepackage{listings}
\usepackage{xcolor}
\definecolor{codegreen}{rgb}{0,0.6,0}
\definecolor{codegray}{rgb}{0.5,0.5,0.5}
\definecolor{codepurple}{rgb}{0.58,0,0.82}
\definecolor{backcolour}{rgb}{0.95,0.95,0.92}

\lstdefinestyle{mystyle}{
    backgroundcolor=\color{backcolour},   
    commentstyle=\color{codegreen},
    keywordstyle=\color{magenta},
    numberstyle=\tiny\color{codegray},
    stringstyle=\color{codepurple},
    basicstyle=\ttfamily\footnotesize,
    breakatwhitespace=false,         
    breaklines=true,                 
    captionpos=b,                    
    keepspaces=true,                 
    numbers=left,                    
    numbersep=5pt,                  
    showspaces=false,                
    showstringspaces=false,
    showtabs=false,                  
    tabsize=2
}

\lstset{style=mystyle}

\usepackage[style=verbose]{biblatex}
\usepackage{fancyhdr}
\usepackage{graphicx,caption,subcaption,color,float} %Graphics stuff
\usepackage{hyperref,amssymb,amsmath, amsfonts, amsthm, enumerate, bm}
\usepackage{placeins, cancel, wrapfig, xcolor, array, multirow, booktabs, algorithm, algpseudocode} 
\usepackage[margin=0.9in]{geometry}
\usepackage{ulem}
\graphicspath{ {figs/} }
\bibliography{references}

% you may include other packages here (next line)
\usepackage{enumitem}
\usepackage{dirtytalk}

%----- you must not change this -----------------
\topmargin -1.0cm
\textheight 23.0cm
\parindent=0pt
\parskip 1ex
\renewcommand{\baselinestretch}{1.1}
\pagestyle{fancy}
\renewcommand{\theenumi}{\Alph{enumi}}
\makeatletter
\newcommand{\distas}[1]{\mathbin{\overset{#1}{\kern\z@\sim}}}%
\newsavebox{\mybox}\newsavebox{\mysim}
\newcommand{\distras}[1]{%
  \savebox{\mybox}{\hbox{\kern3pt$\scriptstyle#1$\kern3pt}}%
  \savebox{\mysim}{\hbox{$\sim$}}%
  \mathbin{\overset{#1}{\kern\z@\resizebox{\wd\mybox}{\ht\mysim}{$\sim$}}}%
}
\makeatother
%----------------------------------------------------

% enter your details here----------------------------------
\lhead{}
\chead{}
\rhead{}
\lfoot{}
\cfoot{}
\rfoot{}
\setlength{\fboxrule}{4pt}\setlength{\fboxsep}{2ex}
\renewcommand{\headrulewidth}{0.4pt}
\renewcommand{\footrulewidth}{0.4pt}


\title{Homework 1}
\author{Jiaxin XU}

\begin{document}

\maketitle

\textbf{Problem 1:} Katz centrality

\[ \textbf{c}_{Katz} = \beta(\textbf{I}-\alpha \textbf{A})^{-1} \textbf{1}. \]

$\alpha$ is a positive constant. When we let $\alpha \to 0$, then all the vertices have the same centrality $\beta$ As we increase $\alpha$ from $0$, the centrality calculated will increase and then comes to a divergence point, where $(\textbf{I}-\alpha \textbf{A})^{-1}$ diverges. That is when

\[ \det(\textbf{I}-\alpha \textbf{A}) = \det(\textbf{A} - \alpha ^{-1}\textbf{I} ) = 0.\]

As $\alpha$ increases, the determinant first crosses $0$ when $\alpha = 1/k_{1}$, where $k_{1}$ is the largest eigenvalue of $\textbf{A}$. Therefore, when $\alpha$ is less than $1/k_{1}$ , the expression for Katz centrality will converge.
\clearpage


\textbf{Problem 2:} Use "walk" to compute the total number of common neighbors $|N(v_{i})\cap N(v_{j})|$ between nodes $v_{i}$ and $v_{j}$.

\[|N(v_{i})\cap N(v_{j})| = N_{ij}^{(2)}, \]
where $ N_{ij}^{(2)}$ is the number of walks of length 2 from $v_{i}$ to $v_{j}$,  
\[ N_{ij}^{(2)} = \sum_{k=1}^{n} A_{ik}A_{kj} = [A^{2}]_{ij} ,\]
and $A$ is the adjacency matrix.

\clearpage

\textbf{Problem 3:} See Appendix for code.
The similarity plot is shown as follows:
\begin{figure}[h]
\includegraphics[width=8cm]{"Figure_1.png"}
\centering
\caption{Jaccard's similarity between "Ginori" family and other families in the Florentine Families graph, edge colored by the corresponding similarity values.}
\end{figure}

\clearpage

\textbf{Appendix}
\begin{lstlisting}[language=Python, caption=Python code]

"""
ACMS 80770-03: Deep Learning with Graphs
Instructor: Navid Shervani-Tabar
Fall 2022
University of Notre Dame

Homework 1: Programming assignment
"""

from operator import le
from platform import node
import networkx as nx
import matplotlib.pyplot as plt
import numpy as np
from networkx.algorithms import bipartite
from networkx.generators.random_graphs import erdos_renyi_graph
import copy


# -- Initialize graphs
seed = 30
G = nx.florentine_families_graph()
nodes = G.nodes()
layout = nx.spring_layout(G, seed=seed)


# -- compute jaccard's similarity
"""
    This example is using NetwrokX's native implementation to compute similarities.
    Write a code to compute Jaccard's similarity and replace with this function.
"""
# pred = nx.jaccard_coefficient(G)
def my_jaccard_similarity(G):
    nodes = list(G.nodes()) # the node names list
    A = nx.to_numpy_array(G)
    # matrix of total number of shared neighbors (intersection)
    A_cap = np.matmul(A,A) 
    # matrix of total number of neighbors (union)
    A_cup = np.zeros_like(A) 
    for i in range(len(A)):
        for j in range(len(A)):
            A_cup[i][j] = sum(A[i])+sum(A[j])-A_cap[i][j]
            
    # Jaccard's similarity matrix
    S = A_cap/A_cup 

    return ((nodes[i],nodes[j],S[i][j]) for i in range(len(A)) for j in range(len(A)))

pred = my_jaccard_similarity(G)


# -- keep a copy of edges in the graph
old_edges = copy.deepcopy(G.edges())

# -- add new edges representing similarities.
new_edges, metric = [], []
for u, v, p in pred:
    G.add_edge(u, v)
    print(f"({u}, {v}) -> {p:.8f}")
    new_edges.append((u, v))
    metric.append(p)

# -- plot Florentine Families graph
nx.draw_networkx_nodes(G, nodelist=nodes, label=nodes, pos=layout, node_size=600)
nx.draw_networkx_edges(G, edgelist=old_edges, pos=layout, edge_color='gray', width=4)

# -- plot edges representing similarity
"""
    This example is randomly plotting similarities between 8 pairs of nodes in the graph. 
    Identify the ”Ginori”
"""
## Identify the ”Ginori”
Ginori_edge_ls = []
Ginori_metric_ls = []
for i in range(len(new_edges)):
    if new_edges[i][0] == 'Ginori' and new_edges[i][1] != 'Ginori':
        Ginori_edge_ls.append(new_edges[i])
        Ginori_metric_ls.append(metric[i])
## plot
ne = nx.draw_networkx_edges(G, edgelist=Ginori_edge_ls, pos=layout, edge_color=np.asarray(Ginori_metric_ls), width=4, alpha=0.7)
plt.colorbar(ne)
plt.axis('off')
plt.show()


\end{lstlisting}
\clearpage

\end{document}